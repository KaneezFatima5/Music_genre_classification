\documentclass{report}
\usepackage{algorithm}
\usepackage{algpseudocode}
% 2023-10-21 modified by Simon King (Simon.King@ed.ac.uk)  
% 2024-01 modified by TPC Chairs of Interspeech 2024  
% 2024-10 modified by Antoine Serrurier for Interspeech 2025
% 2024-12 modified by TPC Chairs of Interspeech 2025

% **************************************
% *    DOUBLE-BLIND REVIEW SETTINGS    *
% **************************************
% Comment out \interspeechcameraready when submitting the 
% paper for review.
% If your paper is accepted, uncomment this to produce the
%  'camera ready' version to submit for publication.

 \interspeechcameraready 


% **************************************
% *                                    *
% *      STOP !   DO NOT DELETE !      *
% *          READ THIS FIRST           *
% *                                    *
% * This template also includes        *
% * important INSTRUCTIONS that you    *
% * must follow when preparing your    *
% * paper. Read it BEFORE replacing    *
% * the content with your own work.    *
% **************************************

% title here must exactly match the title entered into the paper submission system
\title{Random Forests as a Tool for Effective Fraud Detection in Financial Systems}

% the order of authors here must exactly match the order entered into the paper submission system
% note that the COMPLETE list of authors MUST be entered into the paper submission system at the outset, including when submitting your manuscript for double-blind review
\author[affiliation={1}]{Sathvik}{Quadros}
\author[affiliation={1}]{Jacob}{Findley}

%The maximum number of authors in the author list is 20. If the number of contributing authors is more than this, they should be listed in a footnote or the acknowledgement section.

% if you have too many addresses to fit within the available space, try removing the "\\" newlines
\affiliation{Dept of Computer Science}{University of New Mexico}{USA}
\email{sjquadros2004@unm.edu, jfindley@unm.edu}
\keywords{random forest, decision tree, information gain, fraud detection, financial systems, artificial intelligence, machine learning}

\newcommand{\blue}[1]{\textcolor{blue}{#1}}
\usepackage{comment}
\usepackage{soul}
\begin{document}

\maketitle

% the abstract here must exactly match the abstract entered into the paper submission system
\begin{abstract}
    
    % 1000 characters. ASCII characters only. No citations.
     \textbf{ Abstract - Financial systems must deal with hundreds of millions of transactions per day. While many of these are legitimate, a common problem financial systems must deal with is stopping adversaries attempting to commit fraud. Failing to detect fraud could lead to the loss of a significant amount of capital, and cause long lasting damage to the organizations affected. In this paper, we will discuss a machine learning approach to determine whether a specific transaction shows signs of being fraudulent. We implement and train a random forest consisting of many decision trees and evaluate their performance over several parameters such as attribute-selection criteria and confidence level. The experiments show that after all the training, the random forest tends to perform relatively well with an accuracy of 82.3\%, given the inherent nature of fraudulent transactions being to disguise themselves as being legitimate.}
\end{abstract}

\section{Introduction}

The field of machine learning is concerned with the question of how to construct
computer programs that automatically improve with experience. It involves training a model on a large set of examples to extract generalized knowledge that can be extrapolated on. It requires a well specified task, a measure of performance, and plenty of training examples [1].

The classification of fraudulent and legitimate transactions is a well-specified task. Its measure of performance is the accuracy of its predictions. A large, but highly unbalanced set of instances is available, totaling at around 591k instances. 20\% of these instances are reserved for the testing set, with the remaining 472k instances used for training. The large number of samples make machine learning a viable approach for solving the task of classification, and a random forest in particular is suitable as the data is represented as attribute-value pairs and the target function has discrete output values [1].

\section{Design and Implementation}

Authors are encouraged to describe how their work relates to prior work by themselves and by others, and to make clear statements about the novelty of their work. 

\subsection{Data Structures}

All papers submitted to Odyssey 2026 must be original contributions that are not currently submitted to any other conference, workshop, or journal, nor will be submitted to any other conference, workshop, or journal during the review process of Odyssey 2026. Cross-checks with submissions from other conferences will be carried out to enforce this rule.

\subsection{ID3 Algorithm}

All (co-)authors must be responsible and accountable for the work and content of the paper, and they must consent to its submission. Generative AI tools cannot be a co-author of the paper. They can be used for editing and polishing manuscripts, but should not be used for producing a significant part of the manuscript.

\begin{algorithm}
	\caption{id3(An algorithm with caption)}\label{alg:cap}
	\begin{algorithmic}
		\Require $n \geq 0$
		\Ensure $y = x^n$
		\State $y \gets 1$
		\State $X \gets x$
		\State $N \gets n$
		\While{$N \neq 0$}
		\If{$N$ is even}
		\State $X \gets X \times X$
		\State $N \gets \frac{N}{2}$  \Comment{This is a comment}
		\ElsIf{$N$ is odd}
		\State $y \gets y \times X$
		\State $N \gets N - 1$
		\EndIf
		\EndWhile
	\end{algorithmic}
\end{algorithm}

\subsection{Attribute Selection}

The theme of Odyssey ~2026 is \textit{Speech beyond words: Trustworthy Identity, Health, Emotion and more}. Odyssey 2026 continues to be fully committed to advancing speech science and technology while meeting new challenges. Please refer to the conference website for further detail.

\subsection{Overfitting and Branch Pruning}

The theme of Odyssey ~2026 is \textit{Speech beyond words: Trustworthy Identity, Health, Emotion and more}. Odyssey 2026 continues to be fully committed to advancing speech science and technology while meeting new challenges. Please refer to the conference website for further detail.

\subsection{Classification}

The theme of Odyssey ~2026 is \textit{Speech beyond words: Trustworthy Identity, Health, Emotion and more}. Odyssey 2026 continues to be fully committed to advancing speech science and technology while meeting new challenges. Please refer to the conference website for further detail.


\section{Experiments}
For all the experiments we decided on a further 90/10 split of the instances selected for training, using 10\% of them instances for validation.

The page layout should match with the following rules. A highly
recommended way to meet these requirements is to use one of the
templates provided and to check details against this example
file. Do not modify the template layout! Do not reduce the line spacing!

If for some reason you cannot use any of the templates, please
follow these rules as carefully as possible, or contact the
organizers at \mbox{$<$info@odyssey2026.org$>$} for further
instructions.

\subsection{Basic layout features}

\begin{itemize}
\item Proceedings will be printed in A4 format. The layout is
designed so that the papers, when printed in US Letter format, will include
all material but the margins will not be symmetric. PLEASE TRY TO MAKE YOUR
SUBMISSION IN A4 FORMAT, if possible, although this is not an
absolute requirement.
\item Two columns are used except for the title part and possibly for large figures that may need a full page width.
\item Left margin is \SI{20}{\milli\metre}. 
\item Column width is \SI{80}{\milli\metre}. 
\item Spacing between columns is \SI{10}{\milli\metre}.
\item Top margin is \SI{25}{\milli\metre} (except for the first page which is \SI{30}{\milli\metre} to the title top).
\item Text height (without headers and footers) is maximum \SI{235}{\milli\metre}.
\item Page headers and footers must be left empty.
\item No page numbers.
\item Check indentations and spacing by comparing to the example PDF file.
\end{itemize}

\subsection{Section headings}

Section headings are centred in boldface with the first word capitalised and the rest of the heading in lower case. Sub-headings appear like major headings, except they start at the left margin in the column. Sub-sub-headings appear like sub-headings, except they are in italics and not boldface. See the examples in this file. No more than 3 levels of headings should be used.

\section{Fonts}

The font used for the main text is Times. The recommended
font size is 9 points which is also the minimum allowed size.
Other font types may be used if needed for special purposes. Remember, 
however, to embed all the fonts in your final PDF file!

LaTeX users: DO NOT USE THE Computer Modern FONT FOR TEXT (Times is
specified in the style file). If possible, make the final
document using POSTSCRIPT FONTS since, for example, equations with
non-PS Computer Modern are very hard to read on screen.



\subsection{Figures}\label{sec:figures}

Figures must be centred in the column or page (if the figure spans both columns). Figures which span 2 columns must be placed at the top or bottom of a page.
Captions should follow each figure and have the format used in Fig.~\ref{fig:speech_production}. 

Figures should preferably be line drawings. If they contain gray
levels or colors, they should be checked to print well on a
high-quality non-color laser printer.
If some figures contain bitmap images, please ensure that their resolution 
is high enough to preserve readability.



\subsection{Tables}\label{sec:tables}

An example of a table is shown in Table~\ref{tab:example}. Somewhat
different styles are allowed according to the type and purpose of the
table. The caption text may be above or below the table. Tables must be legible when printed in monochrome on A4 paper. 

\begin{table}[th]
  \caption{This is an example of a table}
  \label{tab:example}
  \centering
  \begin{tabular}{ r@{}l  r }
    \toprule
    \multicolumn{2}{c}{\textbf{Ratio}} & 
                                         \multicolumn{1}{c}{\textbf{Decibels}} \\
    \midrule
    $1$                       & $/10$ & $-20$~~~             \\
    $1$                       & $/1$  & $0$~~~               \\
    $2$                       & $/1$  & $\approx 6$~~~       \\
    $3.16$                    & $/1$  & $10$~~~              \\
    $10$                      & $/1$  & $20$~~~              \\
    \bottomrule
  \end{tabular}
  
\end{table}



\subsection{Equations}

Equations should be placed on separate lines and numbered. Examples of equations are given below. Particularly,
%
%\vspace{-3mm}
\begin{equation}
x(t) = s(f_\omega(t))
\label{eq1}
\end{equation}
where \(f_\omega(t)\) is a special warping function
\begin{equation}
f_\omega(t)=\frac{1}{2\pi j}\oint_C \frac{\nu^{-1k}d\nu}
{(1-\beta\nu^{-1})(\nu^{-1}-\beta)}
\label{eq2}
\end{equation}
A residue theorem states that
\begin{equation}
\oint_C F(z)dz=2 \pi j \sum_k Res[F(z),p_k]
\label{eq3}
\end{equation}
Applying (\ref{eq3}) to (\ref{eq1}),
it is straightforward to see that
\begin{equation}
1 + 1 = \pi
\label{eq4}
\end{equation}
%



\begin{figure}[t]
  \centering
  \includegraphics[width=\linewidth]{figure.pdf}
  \caption{Schematic diagram of speech production.}
  \label{fig:speech_production}
\end{figure}

\subsection{Page numbering}

Final page numbers will be added later to the document electronically. {\em
Please do not include any headers or footers!}

\subsection{Style}

Manuscripts must be written in English. Either US or UK spelling is acceptable (but do not mix them).

\subsubsection{References}
\label{section:references}

It is ISCA policy that papers submitted should refer to peer-reviewed publications. References to non-peer-reviewed publications (including public repositories such as arXiv, Preprints, and HAL, software, and personal communications) should only be made if there is no peer-reviewed publication available, should be kept to a minimum, and should appear as footnotes in the text (i.e., not listed in the References).

References should be in standard IEEE format, numbered in order of appearance, for example \cite{Davis80-COP} is cited before \cite{Rabiner89-ATO}. For longer works such as books, provide a single entry for the complete work in the References, then cite specific pages \cite[pp.\ 417--422]{Hastie09-TEO} or a chapter \cite[Chapter 2]{Hastie09-TEO}. Multiple references may be cited in a list \cite{Smith22-XXX, Jones22-XXX}.

\subsubsection{International System of Units (SI)}

Use SI units, correctly formatted with a non-breaking space between the quantity and the unit. In \LaTeX\xspace this is best achieved using the \texttt{siunitx} package (which is already included by the provided \LaTeX\xspace class). This will produce
\SI{25}{\milli\second}, \SI{44.1}{\kilo\hertz} and so on.




\section{Submissions}

Information on how and when to submit your paper is provided on the conference website.

\subsection{Manuscript}

Authors are required to submit a single PDF file of each manuscript. The PDF file should comply with the following requirements: (a) no password protection; (b) all fonts must be embedded; and (c) text searchable (do ctrl-F and try to find a common word such as ``the''). The conference organisers may contact authors of non-complying files to obtain a replacement. Papers for which an acceptable replacement is not provided in a timely manner will be withdrawn.

\subsubsection{Embed all fonts}

It is \textit{very important} that the PDF file embeds all fonts!  PDF files created using \LaTeX, including on \url{https://overleaf.com}, will generally embed all fonts from the body text. However, it is possible that included figures (especially those in PDF or PS format) may use additional fonts that are not embedded, depending how they were created. 

On Windows, the bullzip printer can convert any PDF to have embedded and subsetted fonts. On Linux \& MacOS, converting to and from Postscript will embed all fonts:
\\

\noindent\textsf{pdf2ps file.pdf}\\
\noindent\textsf{ps2pdf -dPDFSETTINGS=/prepress file.ps file.pdf}





\section{Discussion}

Authors must proofread their PDF file prior to submission, to ensure it is correct. Do not rely on proofreading the \LaTeX\xspace source or Word document. \textbf{Please proofread the PDF file before it is submitted.}

\section{Acknowledgements}


\ifinterspeechfinal
     The Odyssey 2026 organisers
\else
     The authors
\fi
would like to thank ISCA and the organising committees of past Interspeech conferences for kindly providing the previous version of this template.


\bibliographystyle{IEEEtran}
\bibliography{bibliography}

\end{document}